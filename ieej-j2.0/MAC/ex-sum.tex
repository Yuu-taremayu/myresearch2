%%% 
%% v3.0 [2011/03/08]
\documentclass[ExtendedSummary]{ieej}
%\usepackage{graphicx}
%\usepackage{latexsym}
\usepackage[varg]{txfonts}

\title{Example of ``Extended Summary''}
\authorlist{%
 \authorentry{Taro Denshi}{m}{DUT}
 \authorentry{Hanako Denki}{n}{DUH}
}
\affiliate[DUT]
 {Denshi University, taro@denshi.ac.jp}
\affiliate[DUH]
 {Denki University, hanako@denki.ac.jp}

\begin{document}
\begin{keyword}
Extended Summary, Class file, \LaTeXe
\end{keyword}
\maketitle

Here is the template of the ``Extended Summary''. 

\onelineskip

\begin{verbatim}
\documentclass[ExtendedSummary]{ieej}
\title{title}
\authorlist{%
 \authorentry{name}{membership}{label}
}
\affiliate[label]
 {short affiliate, E-mail address}
\begin{document}
\begin{keyword}
keywords
\end{keyword}
\maketitle
 ...
\end{document}
\end{verbatim}

\onelineskip

\begin{itemize}
\item
``\texttt{ExtendedSummary}'' must be specified 
as an option of \verb/\documentclass/.

\item
A title is assigned in \verb/\title/. 
You may use \verb/\\/ to start a new line in a long title. 

\item
The outputs of authors' names, memberships, affiliates and E-mail addresses 
are automatically generated by using the \verb/\authorlist/ 
and \verb/\authorentry/ commands, 
combining with \verb/\affiliate/ commands (see below). 

The \verb/\authorentry/ command must be described as 
an argument of the \verb/\authorlist/ command. 

The \verb/\authorentry/ command has three arguments. 

\noindent
\verb/\authorentry{/\textit{name}%
 \verb/}{/\textit{membership}%
  \verb/}{/\textit{label}%
   \verb/}/

For example, they could be typesetted as follows: 
\begin{verbatim*}
\authorlist{%
\authorentry{Taro Denshi}{m}{DUT}
\authorentry{Hanako Denki}{n}{DUH}
}
\end{verbatim*}

\begin{itemize}
\item
The second argument is specified by one letter 
out of seven letters (m, a, s, l, n, h, S), 
each one indicating the membership of authors 
as the following table shows. 

\onelineskip

\begin{center}
\begin{small}
\begin{tabular}{lll}
\hline
\texttt{m} & Member\rule{0mm}{3.5mm}\\
\texttt{a} & Associate\\
\texttt{s} & Student Member\\
\texttt{l} & Life Member\\
\texttt{n} & Non-member\\
\texttt{h} & Honorary Member\\
\texttt{S} & Senior Member\\
\texttt{f} & Fellow\\
\hline
\multicolumn{3}{p{65mm}}{\footnotesize 
 the left column is letters to be specified. 
 the right column is membership to be generated. 
 }
\end{tabular}
\end{small}
\end{center}

\onelineskip

No extra spaces may be added between a letter and a brace. 
\verb*/{m}/ and \verb*/{m }/ are regarded as different. 
The latter will not generate ``Member''. 

\item
The third argument is assigned by the label of the author's affiliate 
and E-mail address corresponding to the label 
of the \verb/\affiliate/ command. 
For example, an abbreviation for university, institute or company 
can be given. 
\end{itemize}

\item
An author's affiliate and E-mail address are described 
in the \verb/\affiliate/ command as follows. 

\verb/\affiliate[/\textit{label}%
 \verb/]{/\textit{short affiliate}, \textit{E-mail address}%%
  \verb/}/

The first argument ``\textit{label}'' must be the same as 
the 3rd argument of the \verb/\authorentry/ command. 
The second argument is assigned by both the author's short affiliate 
and E-mail address. 

You may use \verb/\\/ to fold a long affiliate and E-mail address. 

\item
The text of the keywords is described 
in the \texttt{keyword} environment. 

\item
The \verb/\maketitle/ command must come after those commands 
before the main text begins. 
\end{itemize}

\end{document}
